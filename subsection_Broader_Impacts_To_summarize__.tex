\subsection*{Broader Impacts}

To summarize, I propose to develop an open-source platform for the crowd-sourced image segmentation of brain MRI data, under the guidance of Ariel Rokem and Jason Yeatman at the eScience Institute and the University of Washington Institute for Neuroengineering. Through gamification, piece-wise exposure, and machine learning, I plan to engage a large user base across a variety of image segmentation tasks. Example applications include parcellating gray and white matter in a low contrast image where traditional segmentation algorithms fail, and delineating multiple sclerosis lesions which usually requires trained neuroradiologists. For a particular application, the Yeatman Lab at UW is collecting a large, longitudinal MRI  dataset on children undergoing an intensive learning program, with the goal of determining how experience shapes brain development. The segmentation data from the MindGames platform can be used to 1) define the typical timecourse of cortical changes by examining gray/white matter volumes from segmentation, 2) construct normative developmental curves in order to detect abnormalities, and 3) study how learning shapes brain development by analyzing quantitative MR intensities within the gray and white matter. The MindGames platform will help researchers by improving the precision of segmentation measures without advanced computer science expertise, but will also engage, educate and excite the public and help advance cutting edge neuroscience research. 

%akeshavan_end_stitch
