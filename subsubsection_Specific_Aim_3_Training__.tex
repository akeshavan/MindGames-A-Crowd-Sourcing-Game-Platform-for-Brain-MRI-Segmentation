

\subsection*{Specific Aim 3: Training and Gamification}

For individuals with minimal neuroamatomy knowledge, the difficulty of manual neuroimaging segmentation will depend on the contrast of the image as well as the location/complexity of the target structure. An example of an easy task would be the segmentation of brain tissue from non-brain tissue, whereas a more difficult task would be the segmentation of multiple sclerosis lesions. This Aim will address simple as well as challenging problems through varying levels of training and rewards. A web application will be developed that hooks into the server developed in Aim 1. The app will include an in-browser brain editor (similar to the Mindcontrol application \cite{keshavan2016mindcontrol}), a reward structure and a scoreboard for the top users, and an optional link to the Amazon Turk engine, where users can be paid (in micro-payments) for completing micro-tasks. Initially, the user will only be presented with training tasks until they reach an adequate accuracy score. Next, the training tasks will be interspersed with new tasks, in order to detect performnce drift. The frequency of training tasks will increase based on the researcher's specification of task difficulty. The reward structure will be based on 1) how well the user edits training data, 2) how well the user segmentations match those of other users, and 3) how many voxels are edited by the user. The time spent on the task along with the number of edited voxels will also be used to validate whether or not the user completed the task with some thought. For example, a user's score would be penalized if a large number of voxels were edited too quickly for a difficult task. I plan to collaborate with the data scientists at the eScience Institute to build an intuitive, fun, and engaging crowd-sourcing user interface on the Amazon Turk Platform.