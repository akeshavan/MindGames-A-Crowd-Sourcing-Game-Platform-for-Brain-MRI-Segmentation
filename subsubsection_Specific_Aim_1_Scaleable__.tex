\subsection*{Specific Aim 1: Scaleable and Secure Micro-Tasks}

This aim will address two key challenges: 1) Partitioning 3D data into microtasks that keep data private, 2) serving micro-tasks at scale. While there are many large-scale open-source data collection efforts, many datasets are kept private within research institutions due to  IRB restrictions, so presenting a full 3D MRI volume to the public would be a violation. Serving smaller "chunks" of data serves two purposes: it allows us to keep data private (because you cannot see the whole brain), and it reduces the fatigue of non-experts (because you only need to edit a small section), which enables us to engage a larger user base. A scaleable server will be implemented on a commercial cloud computing platform, with an API that allows researchers to upload MRI micro-tasks to the server database, and serves microtasks to users. Researchers will be asked to provide the following to the API: 1) an initial segmentation file from an automated algorithm 2) any original images (T1, T2, PD) that users need to properly edit the segmentation 3) directions on how the images should be sliced into microtasks (including the slicing plane and the number of slices). Additionally, researchers must provide a validation dataset, which includes "correctly" segmented images, which will be used to train non-experts in aim 3. The resources and faculty at the eScience Institute will help me implement state-of-the-art database and cloud computing technologies in order to increase the delivery of micro-tasks to "citizen-scientists."