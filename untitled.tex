\section*{Introduction}

Advancements in MRI technology and image segmentation algorithms have enabled researchers to begin to understand the mechanisms of healthy brain development \cite{giedd1999brain} and neurological disorders, such as multiple sclerosis \cite{bakshi2008mri}. Due to the large variability of brain morphology in the population, increasingly large sample sizes are needed to answer important/interesting biomedical questions. In order to make sense of all the information, automated algorithms were developed to reduce information-rich 3D MRI images to 1 dimensional summary metrics describing tissue properties, such as total gray matter volume, that are easier to understand. Automated  segmentation algorithms save a lot of time compared to manual human inspection, but lack the advanced visual system of humans. As a result, they often make systematic errors, especially for brains with pathology or early in development. I think that data science can help answer neuroscience research questions using a crowd-sourcing strategy, where errors from automatic quantification can be resolved efficiently by non-expert "citizen-scientists".

Crowd sourcing has been successful in many other disciplines, including mathematics, astronomy, and biochemistry \cite{wiggins2011conservation}. Recently, over 200,000 "citizen-neuroscientists"  from over 147 countries, helped identify neuronal connections in a mouse retina, through the Eyewire game \cite{kim2014space}. This crowd-sourced game led to a new understanding of how mammalian retinal cells detect motion. I propose to implement 3 key features of the EyeWire paradigm to adapt it for multiple neuroimaging use cases. First, by breaking up the problem into smaller "micro-tasks", scientists are able to access a much larger user-pool of non-experts \cite{kittur2008crowdsourcing}. Second, machine learning algorithms were trained to help with the task, which improved the speed of manual neuronal tracing and validated non-expert input. Lastly, EyeWire transformed a dull, monotonous task for experts into a fun, competitive game that trained non-experts and acquired valuable scientific data. I propose to implement these features to create an open-source platform for efficiently crowd-sourcing brain tissue classification problems in order to answer neuroscience research questions with more precision.

\section*{Specific Aims}
\begin{compactenum}
\item \textbf{Scaleable and Secure Micro-Tasks}: A scaleable database system and server backend that keeps data private by dividing it into small "micro-tasks"
\item \textbf{Learning by Example}: Machine learning algorithm that learns from human curation to improve efficiency of manual tasks
\item \textbf{Training through Gamification}: User interface that trains users to solve a specific problem, and keeps them engaged through a reward system 
\end{compactenum}

