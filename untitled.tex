\section*{Rationale}

Advances in MRI technology and image segmentation algorithms have enabled researchers to begin to understand the mechanisms of healthy brain development \cite{giedd1999brain} and neurological disorders, such as multiple sclerosis \cite{bakshi2008mri}. Due to the large variability of brain morphology, increasingly large sample sizes are needed to answer important biomedical questions. Automated algorithms have been developed to reduce information-rich 3D MRI images to 1-dimensional summary measures that describe tissue properties and are easy to interpret, such as total gray matter volume. Automated  segmentation algorithms save a lot of time compared to manual human inspection, but lack the advanced visual system of humans. As a result, they often make systematic errors, especially when analyzing brains with pathology, or those in the early stages of development. Data science is poised to help answer neuroscience research questions using a crowdsourcing strategy, where errors from automatic quantification can be resolved efficiently by non-expert "citizen-scientists".

Crowdsourcing has been successful in many other disciplines \cite{wiggins2011conservation}, including mathematics \cite{cranshaw2011polymath}, astronomy \cite{lintott2008galaxy}, and biochemistry \cite{eiben2012increased} . Recently, over 200,000 "citizen-neuroscientists"  from over 147 countries helped identify neuronal connections in a mouse retina through the Eyewire game \cite{kim2014space}. This crowdsourced game led to a new understanding of how mammalian retinal cells detect motion. I propose to implement three key features of the EyeWire paradigm and adapt them for the segmentation of MRI data. First, by breaking up the problem into smaller "micro-tasks", scientists were able to access a much larger user-pool of non-experts \cite{kittur2008crowdsourcing}. Second, machine learning algorithms were trained to help with the task, which improved the speed of manual neuronal tracing and validated non-expert input. Lastly, EyeWire transformed a dull, monotonous task for experts into a fun, competitive game that trained non-experts and acquired valuable scientific data. I propose to implement these features to create an open-source platform for efficiently crowdsourcing brain tissue classification problems in order to answer neuroscience research questions with more precision.

\section*{Specific Aims}
\begin{compactenum}
\item \textbf{Scaleable and Secure Micro-Tasks}: A scaleable database system and server backend that keeps data private by dividing it into small "micro-tasks"
\item \textbf{Learning by Example}: Machine learning algorithm that learns from human curation to improve efficiency of manual tasks
\item \textbf{Training through Gamification}: User interface that trains users to solve a specific problem, and keeps them engaged through a reward system 
\end{compactenum}

