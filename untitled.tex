%akeshavan_start_stitch
\subsection*{Rationale}

Advances in MRI technology and image segmentation algorithms have enabled researchers to begin to understand the mechanisms of healthy brain development \cite{giedd1999brain} and neurological disorders, such as multiple sclerosis \cite{bakshi2008mri}. Due to the wide variability of brain morphology, coupled with a pathological process in the case of neurological disorders, increasingly large sample sizes are necessary to confidently answer the progressively complex biomedical questions the research community is interested in. Automated algorithms have been developed to reduce information-rich 3D MRI images to 1-dimensional summary measures that describe tissue properties and are easy to interpret, such as total gray matter volume. Automated  segmentation algorithms save considerable time, compared to manual human inspection, but lack the advanced visual system of humans. As a result, these algorithms often make systematic errors, especially when analyzing brains with pathology or those in the early stages of development. Data science is poised to facilitate complex neuroscience research by fusing a crowdsourcing strategy with machine learning methods; automatic quantification can perform the bulk of the work efficiently and errors can be resolved by non-expert "citizen-scientists" with the advantage of the human visual system.

Crowdsourcing has been successful in many other disciplines \cite{wiggins2011conservation}, including mathematics \cite{cranshaw2011polymath}, astronomy \cite{lintott2008galaxy}, and biochemistry \cite{eiben2012increased} . Recently, over 200,000 "citizen-neuroscientists"  from over 147 countries helped identify neuronal connections in a mouse retina through the Eyewire game \cite{kim2014space}. This crowdsourced game led to a new understanding of how mammalian retinal cells detect motion. I propose to implement three key features of the EyeWire paradigm and adapt them for the segmentation of MRI data. First, by breaking up the problem into smaller "micro-tasks", Eyewire scientists were able to access a much larger user-pool of non-experts. In a similar vein, 3D MRI data can be divided into 2D slices to be segmented by users. Second, machine learning algorithms were trained to help with the task, which improved the speed of manual neuronal tracing and validated non-expert input in the Eyewire game. Deep learning methods have already shown to be successful at segmenting MRI data, and similar models could be built to support manual segmentation. Lastly, EyeWire transformed a dull, monotonous task for experts into a fun, competitive game that trained non-experts and acquired valuable scientific data. The University of Washington is an ideal place to develop a similar game platform for MRI segmentation, using the resources at the Center for Game Science, led by Zoran Popovic. I propose to create an open-source platform for efficiently crowdsourcing brain tissue classification problems in order to answer neuroscience research questions with more precision.

\subsection*{Specific Aims}
\begin{compactenum}
\item \textbf{Scaleable and Secure Micro-Tasks}: A scaleable database system and server backend that keeps data private by dividing it into small "micro-tasks"
\item \textbf{Learning by Example}: Machine learning algorithm that learns from human curation to improve efficiency of manual tasks
\item \textbf{Training through Gamification}: User interface that trains users to solve a specific problem, and keeps them engaged through a reward system 
\end{compactenum}

