
\subsection*{Specific Aim 2: Learning by Example}
This Aim will address three challenges: 1) Resolve user input to create a final 3D volume, 2) Prioritize serving micro-tasks based on user consensus and 3) Predict the user-edited segmentation image. To reconstruct the micro-tasks back into a 3D image, a weighted consensus map will be computed, based on how accurately each user performed edits on training data. Micro-tasks with lower consensus scores will be served more frequently to users, until the consensus is high. Participants will also be scored based on how well their segmentations match with other users on the same image, and this will be used to reward users in Aim 3. Finally, improving automated segmentation algorithms based on human input will save time and reduce the number of editors assigned to each micro-task. For example, a dataset of 100 3D volumes could be broken into 20,000 patches, each of which would need to be manually edited. Alternatively, convolutional neural networks (CNNs) have been very successful at pattern recognition when trained on similarly large sample sizes, and could reduce the time spent editing each patch. I propose to build a CNN using existing architecture, such as Tensorflow or Theanet, to predict segmentation results, under the guidance of the machine learning experts at the eScience institute. 

